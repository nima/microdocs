\documentclass{microdoc}

\title{Ramanujan, Modular Functions, $\pi$, Oh My! Part I}%
\author{\authb{Robert Smith}{quadricode@gmail.com}}%
\docnum{0x00D}%
\version{0.1}%
\keywords{class polynomial, Hilbert, near-integer, pi, Ramanujan,
  Weber}%
\prereqs{Some knowledge about number theory, polynomials, a little bit
  of complex analysis, and the like. A review of group theory is
  presented.}%
\abstract{This is the first part of a series about so-called
  \emph{near-integers}, irrational numbers that are incredibly close
  to an integer. We expose this phenomenon, and show how it can be
  used to develop formulas to compute $\pi$, and how it's all related
  to a bizarre concept in group theory called the \emph{Monster
    group}. In this part, Part I, we will focus more so on
  near-integers and their relationships in number theory, only briefly
  touching on the latter topics, which will be elucidated in later
  parts.}

\usepackage{amssymb}
\newcommand{\ee}{\ensuremath{\mathrm{e}}}
\newcommand{\ii}{\ensuremath{\mathrm{i}}}
\newcommand{\SL}{\ensuremath{\mathrm{SL}}}
\newcommand{\PSL}{\ensuremath{\mathrm{PSL}}}
\newcommand{\defn}[1]{\textbf{#1}}

\begin{document}
\maketitle

\section{Introduction}
Let $\ee$ be the usual constant approximately equal to
$2.718281828$. One can see that
\begin{equation}
  \label{eq:ramconst}%
  \ee^{\pi\sqrt{163}} = 262537412640768743.99999999999925\ldots
\end{equation}
It is immediately apparent that this number is very, very close to an
integer. This is quite unusual since $\ee$ and $\pi$ are so called
\defn{transcendental numbers}, which means no polynomial whose
coefficients are rational ever have them as a root. Yet $\ee^{\pi\surd
  163}$ is absurdly close to an integer, and all integers are
non-transcendental, or algebraic. \defn{Algebraic numbers} are those
which \emph{can} be represented by the root of some polynomial whose
coefficients are rational. All numbers of the form $\sqrt{n}$ where
$n$ is a rational number are algebraic, because they satisfy $x^2 - n
= 0$, a polynomial of degree $2$ with coefficients $1$ and $-n$.

The (reasonable) skeptic might dismiss \eqref{eq:ramconst} as just a
coincidence, for there are an infinitude of numbers of the form
$\ee^{\pi\surd n}$, and at least one of them is bound to be close to
an integer. To that skeptic, I say
\begin{align}
  \ee^{\pi\sqrt{67}}  &= 5280^3 + 743.9999986\ldots\label{eq:hilbert1}\\
  \ee^{\pi\sqrt{163}} &= 640320^3 + 743.99999999999925\ldots\label{eq:hilbert2}\\
  \intertext{and}
  \ee^{\pi\sqrt{22}} &= 2^6 (1+\sqrt{2})^{12} + 23.99988\ldots\label{eq:weber1}\\
  \ee^{\pi\sqrt{58}} &= 2^6 \left(\frac{5+\sqrt{29}}{2}\right)^{12} + 23.999999988\ldots\label{eq:weber2}\\
  \intertext{and}
  \ee^{\pi\sqrt{42}}  &= 4^4 (21 + 8\sqrt{6})^4 - 104.0000062\ldots\label{eq:raman1}\\
  \ee^{\pi\sqrt{130}} &= 12^4 (323 + 40\sqrt{65})^4 - 104.0000000000012\ldots\label{eq:raman2}
\end{align}
One hopes the skeptic's curiosity has begun to tingle; such
relationships are difficult to label as simple coincidence.

Such a curiosity is well deserved; there is indeed a pattern going
on. The pair of equations \eqref{eq:hilbert1} and \eqref{eq:hilbert2}
are called \defn{near-integers}, and have to do with the concept of
modular functions and Hilbert class polynomials. The latter pairs of
equations are not near-integers, but they are \defn{near-algebraics}
since they are so close to an algebraic number (which can be seen by
the fact that they are approximately an algebraic number\footnote{If
  this isn't obvious, allow me to develop some intuition. Algebraic
  numbers are those composed of the usual arithmetic operations:
  addition, subtraction, multiplication, and division; along with
  exponentiation by an integer or rational. Since $x^{1/2}=\sqrt{x}$,
  algebraic numbers will often contain square roots, and similarly,
  cube roots, etc. This comes as no surprise, since an algebraic
  number is \emph{defined} as a number which is the \emph{root} of a
  polynomial.} plus a near-integer). They are related to the Weber and
Ramanujan class polynomials respectively.

\section{Groups}
\subsection{Review of Groups}
A \defn{group} is a set of objects $S$ along with a binary operation $+$
such that
\begin{itemize}
  \item $a+b$ is in $S$ for all $a,b\in S$,
  \item $a+(b+c) = (a+b)+c$ holds for all $a,b,c$,
  \item there is some element of $S$ called the \defn{identity
    element} $\iota$ such that for all $x\in S$, $\iota+x = x+\iota = x$, and
  \item for all $x\in S$, there exists another element in the group
  $y$ such that $x+y = \iota$. The element $y$ is called the
  \defn{inverse} of $x$, and is sometimes denoted $x^{-1}$.
\end{itemize}
Note that $+$ may not refer to the usual arithmetic addition, but any
possible operator.

A simple example of a group is the set of integers $\mathbb{Z}$ along
with the addition operation $+$. It is obvious that if
$a,b,c\in\mathbb{Z}$, that $a+b$ will also be an integer, that the
``precedence'' of addition doesn't matter (we can add any pair of
integers first, as long as we don't change the order), that the number
$0$ can be added to any number without changing it, and there is a
number $y$ corresponding to every number $x$ such that $x+y=0$,
namely, the negative of $x$.

A more abstract example of a group would be the group of $90^\circ$
rotations and horizontal/vertical/diagonal reflections of a harpoon,
$\upharpoonleft$. Denote a rotation of $90^\circ$ clockwise as $r$. We
can always turn it counter clockwise to get the original harpoon. So
one clockwise and one counterclockwise rotation are
inverses. Therefore, denote the counterclockwise rotation as $r^{-1}$
so $r r^{-1} = \iota$, where $\iota$ means ``no rotation''. We can
write successive rotations as $rrr$ or simply $r^3$, meaning ``three
clockwise rotations''. Immediately, we can see that $r^3 = r^{-1}$.

Denote $h$ as a horizontal flip. So applying $h$ to $\upharpoonleft$
would give $\downharpoonleft$, Let $v$ be a vertical flip, so applying
$v$ to $\upharpoonleft$ gives $\upharpoonright$. Lastly, let $d_1$ be
a diagonal flip, so that applying $d_1$ to $\upharpoonleft$ gives
$\rightharpoondown$. Let $d_2$ be the flip along the other diagonal,
so $d_2$ on $\upharpoonleft$ gives $\leftharpoonup$.

We can see that all of the flips are their own inverse; $h^2=\iota$,
$v^2=\iota$, etc. We also might see that the flips are related to each
other when we use rotations---$rv = d_1$ (note that we perform the
rightmost action first, i.e., $rv$ is a vertical flip, then a
rotation). We can derive a number of other relationships. Most
importantly, for any number of actions we choose, we can always
rewrite it as another action. That is, no matter how we rotate or flip
the harpoon, we will always have a harpoon that requires only one
action\footnote{We regard consecutive rotations as a single action. So
  $r^3$ is a single rotation of $270^\circ$.}. Although the order of
the actions \emph{does} matter, it is easy to see that as long as we
don't change the order, we can perform the actions in any way we
want. Lastly, it is obvious how to make an inverse of any sequence of
actions; just do them in reverse! So since every action is equivalent
to a single action, and precedence does not matter, and we are allowed
a ``do nothing'' action, and we can invert all actions, we have formed
a group. This is commonly known as the \defn{dihedral group} of order $4$,
symbolically written $D_4$.

A \defn{field} is a group $(S,+)$ with an additional operation $\cdot$
satisfying the same rules, along with the fact that for the elements
$a,b,c\in S$, $a\cdot(b+c) = (a\cdot b)+(a\cdot c)$, called the
\defn{distributive property}. The rational numbers, real numbers, and
complex numbers with addition and multiplication are examples of a
field.

\subsection{The Special Linear Group}
The \defn{special linear group} is a group composed of the set of
$n\times n$ matrices whose entries are in some field $F$ and whose
determinant is the multiplicative identity of the field ($1$ in the
case of real or rational numbers). We denote this group as
$\SL_n(F)$. Recall that if an $n\times n$ matrix of real numbers has a
determinant of $1$, we can regard the matrix as a volume (and
orientation) preserving linear transformation in $n$-dimensional
space.

\section{Modular Forms}
\subsection{The Modular Group}
Recall that a \defn{M\"obius transformation} is a function
$f:\mathbb{C}\to\mathbb{C}$ where \[f(z) = \frac{az+b}{cz+d}\] for
integers $a,b,c,d$ and $ad-bc\neq 0$. We can represent such
transformations as a matrix
\[
\begin{pmatrix}
a & b\\
c & d
\end{pmatrix}
\]
such that $ad-bc\neq 0$. One should recognize that
\[
\begin{vmatrix}
  a & b\\
  c & d
\end{vmatrix}
= ad-bc\text,
\]
so the M\"obius transformations can be regarded as the set of $2\times
2$ integer matrices with non-zero determinants.

If we impose the restriction that the determinants need to be $1$,
then we have formed $\SL_2(\mathbb{Z})$. However, if such matrices
represent M\"obius transformations, then
\[
-
\begin{pmatrix}
a & b\\
c & d
\end{pmatrix}
\equiv
\begin{pmatrix}
a & b\\
c & d
\end{pmatrix}
\]
since
\[
\frac{(-a)z+(-b)}{(-c)z+(-d)} = \frac{-1}{-1} \cdot \frac{az+b}{cz+d} = \frac{az+b}{cz+d}\text{.}
\]
Therefore, we actually form a \defn{quotient group}
$\SL_2(\mathbb{Z})/\{1,-1\}$, which means that for all elements $x$ in
the group, $x = -x$. This group is known as the \defn{projective
  special linear group} of order $2$ over the integers, denoted
$\PSL_2(\mathbb{Z})$, and is defined by the aformentioned matrices
with multiplication as the group operator.

In the context of M\"obius transformations, the set of M\"obius
transformations with $ad-bc=1$, along with the composition
operator\footnote{Given two functions $f$ and $g$, the composition
  $f\circ g$ is defined by $(f\circ g)(x) = f(g(x))$.}, forms a group
called the \defn{modular group}, denoted $\varGamma$. If we define
$A:z\mapsto -1/z$ and $B:z\mapsto z+1$, then it can be shown that
$\varGamma$ is generated by $A$ and $B$. Noting that $A^2=(AB)^3=1$,
we have the presentation \[\varGamma \cong \langle A,B \mid
A^2=(AB)^3=1\rangle\text.\] If we generate the group with $A$ and $AB$
instead, then we have $\varGamma\cong C_2*C_3$ where $C_n$ is the
$n$th order cyclic group.

We generalize the notion of a modular group in the following way. Let
$n$ and $m$ be positive integers. Define the modular group
$\varGamma_n(m)$ by
\begin{equation}
  \varGamma_n(m) =
\left\{
\left.
\begin{pmatrix}
a & b\\
c & d
\end{pmatrix}
\in\SL_2(\mathbb{Z})\right\vert
c\equiv 0 \pmod{m}
\right\}\text{.}
\end{equation}
This definition will prove useful later.

\subsection{Modular Functions}
A \defn{modular function} is a complex function $f$ such that
\begin{enumerate}
  \item $f(\tau)$ is equal to its own Taylor series except at isolated
  points (i.e., $f$ is meromorphic),
  \item for every $X\in\PSL_2(\mathbb{Z})$, $f(X\tau) = f(\tau)$, and
  \item the Fourier expansion of $f(\tau)$ is of the form
  $\sum_{k=-n}^\infty a_k \ee^{2k\pi\ii\tau}$ for some $n$.
\end{enumerate}
One can expect such a function to be extremely symmetrical, since it
is ``periodic'' with every transformation in $\varGamma$, since
$\varGamma\cong\PSL_2(\mathbb{Z})$.

We can generalize this by using the generalized notion of a modular
group. A \defn{modular form of weight $k$ of level $m$}, for integers
$k$ and $m$, is a holomorphic function $f$ such that for all $\left(
\begin{smallmatrix}
a & b\\
c & d
\end{smallmatrix}\right)\in\varGamma_0(m)$, and $\Im z > 0$, we
have \[f\left(\frac{az+b}{cz+d}\right)=(cz+d)^k f(z)\] and $f$ is
meromorphic as $z\to +\ii\infty$. It can be seen that such modular
forms have a period of $1$, implying the third point enumerated above.

\section{Elliptic Functions}
An elliptic function is a complex map which is doubly periodic. Let
$\omega_1$ and $\omega_2$ be complex numbers. Then $f$ is an
\defn{elliptic function} if $f(z+\omega_1)=f(z+\omega_2)=f(z)$ for all complex
$z$. From this, we can conclude that
$f(z+m\omega_1+n\omega_2)=f(z)$. If $\{x\in\mathbb{Z}\mid
f(z+x)=f(z)\}$ can be generated entirely by the basis
$\{\omega_1,\omega_2\}$, then we say $\omega_1$ and $\omega_2$ are
\defn{fundamental periods}.

The fundamental periods are not unique, for if we have $\omega_1$ and
$\omega_2$, we can construct the basis $\{a\omega_1+b\omega_2,
c\omega_1+d\omega_2\}$ for integral $a,b,c,d$ such that $ad-bc=1$. As
a consequence, $\left(
\begin{smallmatrix}
  a & b\\c & d
\end{smallmatrix}
\right)$ is an element of the modular group.

\subsection{Weierstrass Elliptic Functions}
The \defn{Weierstrass elliptic function} is an important elliptic function
that is sometimes used to canonicalize all elliptic
functions\footnote{Almost all elliptic functions can be written in
  terms of the Weierstrass function and its derivative}. For periods
$\omega_1$ and $\omega_2$, it is defined by
\begin{equation}
\wp(z; \omega_1, \omega_2) = \frac{1}{z^2} + \sum_{(i,j)\in\mathbb{Z}^2\setminus(0,0)}
    \left[\frac{1}{(z-i\omega_1-j\omega_2)} - \frac{1}{(i\omega_1+j\omega_2)^2}\right]\text.
\end{equation}
If we define $\wp(z;\tau)=\wp(z;1,\tau)$, then we have
$\wp(z;\omega_1,\omega_2)=\wp(z/\omega_1;\omega_2/\omega_1)/\omega_1^2$,
which is expository of its periodicity.

\subsection{Modular Discriminant}
The Laurent series of the Weierstass function is
\[
\wp(z;\omega_1,\omega_2) = \frac{1}{z^2} + \frac{1}{20}g_2 z^2 +
\frac{1}{28}g_3 z^4 + \mathrm{O}(z^6)
\]
for
\[
g_2 = 60 \sum_{i,j}\frac{1}{(i\omega_1+j\omega_2)^{4}}
\]
and
\[
g_3 = 140 \sum_{i,j}\frac{1}{(i\omega_1+j\omega_2)^{6}}\text,
\]
which are called \defn{invariants} of the Weierstrass function.

The Weierstrass function is the inverse of the
function
\[
z\mapsto\int_z^\infty\frac{\mathrm{d}t}{\sqrt{4t^3 -
    g_2t-g_3}}
\] which implies that $[\wp'(z)]^2=4[\wp(z)]^3-g_2\wp(z)-g_3$ (to
verify this differential equation, simply compute poles and compare
the asymptotic expansions). If $y=\wp'(z)$ and $x=\wp(z)$, then we
have $y^2=4x^3-g_2x-g_3$, an \defn{elliptic curve}. This implies that
elliptic curves may be parameterized\footnote{Actually, the
  Weierstrass function lets us map from a torus to the complex plane
  via $z\mapsto (1,\wp(z),\wp'(z))$.} by $\wp$ and $\wp'$.

Since the values of $g_2$ and $g_3$ determine $\omega_1$ and
$\omega_2$, \textit{vice versa}, it is plausible that they may be used
to determine the discriminant of the elliptic curve. Since
$\mathbb{C}$ is a splitting field for $\mathbb{Z}[x]$, we may write
a general elliptic curve as $y^2=x(x-1)(x-\lambda)$. From this we
determine that \[g_2=\frac{\sqrt[3]{4}}{3}(\lambda^2-\lambda+1)\]
and \[g_3=\frac{1}{27}(\lambda+1)(2\lambda^2-5\lambda+2)\text.\] We
find therefore that $\lambda^2(\lambda-1)^2=g_2^3 - 27g_3^2$, which is
called the \defn{modular discriminant}, and is denoted $\Delta$.

One must remember that $g_2$ and $g_3$ are \emph{functions} and
therefore the discriminant is too. One can show that $\Delta(\tau)$ is
a modular form of weight $12$ where $\tau=\omega_2/\omega_1$, called
the \defn{half-period ratio}.

The discriminant of an elliptic curve has a similar role to the
discriminant of a quadratic function---in this case, the discriminant
determines whether or not the curve has a node or a cusp.
\end{document}