%!TEX TS-program = LaTeX
%!TEX encoding = UTF-8 Unicode

\documentclass[namedreferences]{autons}

\newcommand{\theId}{0x004}
\newcommand{\theVersion}{0.4}
\newcommand{\theTitle}{Transitive Noninterference Tutorial}
\newcommand{\theKeywords}{security, assignment}
\newcommand{\theAbstract}{This tutorial is based on the material covered in \cite{vdML5b10}.  For questions 3-5, we make references to judgements and typing systems defined in \cite{SY08}, and elaborated upon in \cite{MD003}}

%%%%%%%%%%%%%%%%%%%%%%%%%%%%%%%%%%%%%%%%%%%%%%%%%%%%%%%%%%%%%%%%%%%%%%%%%%%%%%%%
%. The Preamble...
\def\copyrightline{\copyright\- Nima Talebi 2010}
\newcommand{\theFirstName}{Nima}
\newcommand{\theLastName}{Talebi}
\newcommand{\theEmail}{nima@autonomy.net.au}
\newcommand{\theSubtitle}{Microdocs - Article {\tt\theId}}
\newcommand{\theInstitute}{Autonomy Corporation Pty Ltd}
\newcommand\theAuthor{\theFirstName \theLastName}
%-------------------------------------------------------------------------------
\usepackage[usenames,dvipsnames]{color}
\usepackage[
    pdftitle={\theTitle},
    pdfsubject={\theSubtitle},
    pdfauthor={\theAuthor},
    pdfcreator={\theAuthor},
    pdfkeywords={\theKeywords},
    backref,
    hyperfigures=true,
    final,
    bookmarks=false,
    xetex,
    dvipdfm,
    breaklinks,
    bookmarksopen=true,
    bookmarksnumbered=true,
    colorlinks=true,
    linkcolor=Cyan,
    anchorcolor=Red,
    citecolor=ForestGreen,
    filecolor=Aquamarine,
    menucolor=Orange,
    runcolor=Red,
    urlcolor=LimeGreen
]{hyperref}
\InputIfFileExists{preamble.tex}{}{}

%%%%%%%%%%%%%%%%%%%%%%%%%%%%%%%%%%%%%%%%%%%%%%%%%%%%%%%%%%%%%%%%%%%%%%%%%%%%%%%%
%. The Document...
\begin{document}
\begin{article}
\begin{opening}
    \title{\theTitle}
    \runningtitle{\theTitle}
    \subtitle{\theSubtitle}

    \runningauthor{\theAuthor}
    \author{\theFirstName\ \surname{\theLastName}\email{\theEmail}}
    \institute{\theInstitute}
    \InputIfFileExists{authors.tex}{}{}

    \IfFileExists{motto.tex}{
        \begin{motto}[prose]
            The only thing that never looks right is a rule. There is not in existence a page with a rule on it that cannot be instantly and obviously improved by taking the rule out.  \rightline{George Bernard Shaw, In {\it The Dolphin} (1940)}


        \end{motto}
    }{}

    \begin{abstract}
        \theAbstract
    \end{abstract}
    \keywords{\theKeywords}

    \InputIfFileExists{abbrev.tex}{}{}
    \InputIfFileExists{nomen.tex}{}{}
    \InputIfFileExists{dedication.tex}{}{}
    \InputIfFileExists{classify.tex}{}{}
    \classification{Microdoc}{\theId}
    \classification{Version}{\theVersion}
\end{opening}

%-------------------------------------------------------------------------------
%. Finish off...
\addtocounter{tocdepth}{1}
\tableofcontents

\raggedbottom

\section{\question{1}{5}}
Let the system $\Fancy{M}$ denote the finite state machine which bounds this particular problem space.

Let $\set{D}$ denote the set of security domains (agents), where $\set{D} = \{L, H\}$.  The set $\set{S}$ defines all possible states that $\Fancy{M}$ can hold, and each state is defined by the tuple $(b, l, w) \in \set{S}$. The state tuple components $b, l, w$ are boolean variables;  $b$ and $l$ are $H$ variables, while $w$ is a $L$ variable.

Let the set $\set{A}_H = \{ \lLock, \lUnlock, \lWriteZ, \lWriteO \}$ define the actions that $H$ is capable of executing, while $\set{A}_H = \{ \lWriteZ, \lWriteO \}$ defines the set of actions that $L$ is capable of executing.

The following set of rules then define the dynamics of the state machine $\Fancy{M}$, depicted in figure \autoref{fig:fsm}.  The format of each rule is $s_0 \cdot a = s_1$ followed restrictions if any, where $s_0$ denotes the initial state, $a$ denotes the action, and $s_1$ the final state after the action has been performed.

\begin{figure}[here]
    \centerline{\includegraphics[width=16pc]{fsm.pdf}}
    \caption{The finite state machine $\Fancy{M}$.}
    \label{fig:fsm}
\end{figure}

\begin{subequation}[arabic]
Agent $H$ can perform the following actions with the respective consequences:
\begin{flalign}
    (b, l, w) \cdot \lLock &= (b, 1, w) \quad \forall b, l, w\\
    (b, l, w) \cdot \lUnlock &= (b, 0, w) \quad \forall b, l, w\\
    (b, l, w) \cdot \lWriteZ &= (0, l, w) \Iff l = 1\\
    (b, l, w) \cdot \lWriteO &= (1, l, w) \Iff l = 1\\
    \nonumber
\end{flalign}

Agent $L$ can perform the following actions with the respective consequences:
\begin{flalign}
    (b, l, w) \cdot \lWriteZ &= \left\{
        \begin{array}{l l}
            (0, l, 0) &\quad \Iff l = 0\\
            (b, l, 0) &\text{ otherwise}\\
        \end{array}
    \right.\\
    (b, l, w) \cdot \lWriteO &= \left\{
        \begin{array}{l l}
            (1, l, 1) &\quad \Iff l = 0\\
            (b, l, 1) &\text{ otherwise}\\
        \end{array}
    \right.\\
    \nonumber
\end{flalign}
\end{subequation}

In line with the above rules we define the following observations:
\begin{subequation}[arabic]
\begin{flalign}
    \lobs_L(b, l, w) &= (w)\\
    \lobs_H(b, l, w) &= (b, l)\\
    \nonumber
\end{flalign}
\end{subequation}

\begin{lemma}
Prove that this system $\Fancy{M}$ is secure by showing that there exists an unwinding on this system; That is, define the unwinding and prove that it is an \emph{equivalence relation} satisfying \lOC, \lSC, and \lLR.
\end{lemma}

\vskip0pt
\begin{pf*}{Proof (of Lemma 1)}
\begin{subequation}[arabic]
We begin with the following definition from \cite{vdML5b10}; If some function $f$ exists that satisfies the following requirement, then there will also exist an equivalence relation $\sim$ on $\set{S}$:
\begin{flalign}
    \forall s_1, s_2 \in \set{S}
        \colon s_1 \sim_d s_2 \ \If f(d, s_1) = f(d, s_2) \Where d \in \set{D}\label{eq:q1d1}\\
    \nonumber
\end{flalign}

We assume that $f$ does indeed exist, and intuitively define it as follows:
\begin{flalign}
    f(d, s) = \lobs_d(s) = \left\{
        \begin{array}{l l}
            (w) &\quad \Where d = l\\
            (b, l) &\quad \Where d = h\\
        \end{array}
        \right.\\
    \nonumber
\end{flalign}

The fact that $f$ exists \eqref{eq:q1d1} is actually already proof that the binary relation on $\sim$ is \emph{reflexiv}, \emph{transitiv}, and \emph{symmetric};  however we will explicitly show this for completeness:

First reflexivity:
\begin{flalign*}
    f(s, d) = f(s, d)\\
    \therefore s \sim_d s\\
    \nonumber
\end{flalign*}

Then symmetry:
\begin{flalign*}
    f(s_1, d) = f(s_2, d) &\Rightarrow f(s_2, d) = f(s_1, d)\\
    \therefore s_1 \sim_d s_2 &\Rightarrow s_2 \sim_d s_1\\
    \nonumber
\end{flalign*}

And finally transitivity:
\begin{flalign*}
    f(s_1, d) \leq f(s_2, d) \leq f(s_3, d) &\Rightarrow f(s_1, d) \leq f(s_3, d)\\
    \therefore s_1 \sim_d s_2 \sim_d s_3 &\Rightarrow s_1 \sim_d s_3\\
    \nonumber
\end{flalign*}
\end{subequation}

\begin{subequation}[arabic]
Now, we must satisfy \lOC, \lSC, and \lLR:
\begin{flalign}
    \lOC \colon& \mbox{if } s_1 \sim_d s_2
        \mbox{ then } \lobs_d(s_1) = \lobs_d(s_2)\label{eq:oceq}\\
    \lSC \colon& \mbox{if } s_1 \sim_d s_2
        \mbox{ and } a \in \set{A}
        \mbox{ then } s_1 \cdot a \sim_d s_2 \cdot a \label{eq:sceq}\\
    \lLR \colon& \mbox{if } \ldom(a) \not\rightarrowtail d
        \mbox{ then } s \sim_d s \cdot a \label{eq:lreq}\\
    \nonumber
\end{flalign}

The following table illustrates the agent ($d$), the initial machine state ($s_0$), the action taken by the agent ($a$), the resultant state ($s_1$), followed by $f$ for both agents ($L$ and $H$) and $\ldom(a)$:

\begin{tabular}{ cllccc }
    \hline
    $d$ & $s_0 \cdot a$ & $s_1$ & $f_H$ & $f_L$ & $\ldom_L(a)$\\
    \hline
    $H$ & $(b, l, w).\lLock$   & $(b, 1, w)$ & $(b, 1)$ & $(w)$ & $H$\\
    $H$ & $(b, l, w).\lUnlock$ & $(b, 0, w)$ & $(b, 0)$ & $(w)$ & $H$\\
    $H$ & $(b, 1, w).\lWriteZ$ & $(0, l, w)$ & $(0, 1)$ & $(w)$ & $H$\\
    $H$ & $(b, 0, w).\lWriteZ$ & $(b, l, w)$ & $(b, l)$ & $(w)$ & $H$\\
    $H$ & $(b, 1, w).\lWriteO$ & $(0, l, w)$ & $(0, 1)$ & $(w)$ & $H$\\
    $H$ & $(b, 0, w).\lWriteO$ & $(b, l, w)$ & $(b, l)$ & $(w)$ & $H$\\
    $L$ & $(b, 0, w).\lWriteZ$ & $(0, l, 0)$ & $(0, l)$ & $(0)$ & $L$\\
    $L$ & $(b, 1, w).\lWriteZ$ & $(b, l, 0)$ & $(b, l)$ & $(0)$ & $L$\\
    $L$ & $(b, 0, w).\lWriteO$ & $(1, l, 1)$ & $(1, l)$ & $(1)$ & $L$\\
    $L$ & $(b, 1, w).\lWriteO$ & $(b, l, 1)$ & $(b, l)$ & $(1)$ & $L$\\
    \hline
\end{tabular}
\end{subequation}

Now suppose that $s_1 \sim_L s_2$, then $\lobs_L(s_1) = (w) = \lobs_L(s_2)$, which satisfies \eqref{eq:oceq}.  Also note that if $s_1 \sim_L s_2$, that $\forall a \in \set{A}$ and $\forall d \in \set{D}$, $f_d(s_1 \cdot a) = f_d(s_2 \cdot a)$ and hence \eqref{eq:sceq}.  Finally, notice that for all cases where $\ldom(a) \not\rightarrowtail d$, that $f_d(s) = f_d(s \cdot a)$ and hence \eqref{eq:lreq}.
    \qed
\end{pf*}

\section{\question{2}{5}}
\begin{lemma}
For the \emph{doubling construction} as defined in \cite{vdML5b10}, show that:
\begin{subequation}[arabic]
\begin{flalign}
    \If &s_0' \cdot \alpha = (s, t) \in \Fancy{M}^2
        \Where s, t \in \Fancy{M}\label{eq:q2g1}\\
    \Then &s_0 \cdot \alpha = s \label{eq:q2g2}\\
    \mbox{and } &s_0 \cdot \lpurge_{Low}(\alpha) = t\label{eq:q2g3}\\
    \nonumber
\end{flalign}
\end{subequation}
\end{lemma}

To begin with, some notation:
\begin{subequation}[arabic]
\begin{flalign}
    \Fancy{M} &\colon \mbox{a deterministic state machine}\\
    \Fancy{M}^2 &\colon \mbox{a doubling construct of $\Fancy{M}$}\\
    \set{S} &\colon \mbox{a set of states of the state machine}\\
    s \in \set{S} &\colon \mbox{a particular state}\\
    s_0 \in \set{S} &\colon \mbox{the initial state}\\
    \set{A} &\colon \mbox{a set of actions}\\
    a \in \set{A} &\colon \mbox{a particular action}\\
    \set{D} &\colon \mbox{a set - security domain}\\
    d \in \set{D} &\colon
        \mbox{a particular security domain}\footnotemark[2]\\
    \set{A}^* &\colon \mbox{the infinite set of action sequences}\\
    \alpha \in \set{A}^* &\colon \mbox{an action sequence (ordered)}\\
    \epsilon \in \set{A}^* &\colon \mbox{the empty action sequence}\\
    \rightarrowtail &\colon \label{nonintereq1}
        \mbox{a noninterference policy}\footnotemark[3]\\
    d_1 \rightarrowtail d_2 \label{nonintereq2} &\colon
        \mbox{permitted information flow}\footnotemark[4]\\
    \nonumber
\end{flalign}
\end{subequation}
Equipped with the abovementioned building units, we illustrate how they relate to one another in following rules and constructs. \footnote{We reveal the details of $\Fancy{M}^2$ in due course}
\footnotetext[2]{We may also refer to the security domains represented by $d$, more succinctly as ``agents''.}
\footnotetext[3]{represented as a binary relation}
\footnotetext[4]{information flow permitted from domain $d_1$ to domain $d_2$}

\begin{subequation}[arabic]
\begin{flalign}
    \Fancy{M} &= \langle \set{S}, s_0, \set{A}, \lstep, \lobs, \ldom \rangle \label{eq:dfsm}\\
    \ldom \colon \set{A} &\to \set{D}\label{domeq}\\
    \lpurge \colon \set{A}^* \times \set{D} &\to \set{A}^*\label{purgeeq}\\
    \lstep \colon \set{S} \times \set{A} &\to \set{S}\label{stepeq}\\
    s \cdot a &= s'\label{onestepeq}\\
    s \cdot \alpha &= s''\label{seqofstepeq}\\
    s \cdot \epsilon &= s\label{nullstepeq}\\
    \lobs \colon \set{S} \times \set{D} &\to \set{O}\label{obseq}\\
    \rightarrowtail &\subseteq \set{D} \times \set{D}\label{nonintereq3}\\
    \nonumber
\end{flalign}

Eq. \eqref{domeq} associates each action to an element of the security domain.
\eqref{purgeeq} maps domain only to action permitted to interfere with it.
\eqref{stepeq} depicts a deterministic transition function.
\eqref{onestepeq} depicts a single deterministic transition from $s$ to $s'$ via action $a$.
\eqref{seqofstepeq} depicts state $s''$ reached by performing the sequence of actions $\alpha$ from $s$.
\eqref{nullstepeq} depicts a no-action, no change in state.
\eqref{obseq} maps states to an observation for each security domain.
\eqref{nonintereq3} - see \eqref{nonintereq1} and  \eqref{nonintereq2}.

\end{subequation}

\begin{subequation}[arabic]
Standard model checking verifies properties of a single run of the system, so cannot directly be applied to verify noninterference security, which talks about two runs ($\alpha$ and $\lpurge_d(\alpha)$).  However, we can modify the system so that model checking becomes applicable.

\label{doubling}We will now demonstrate this for the simple case of the \emph{High-Low policy}, which is defined as follows:
\begin{flalign}
    \set{D} &= \{ High, Low \}\\
    Low \rightarrowtail High &\colon \mbox{Low to High information flow permitted}\\
    High \not\rightarrowtail Low &\colon \mbox{High to Low information flow NOT permitted}\\
    \rightarrowtail &= \{ (Low, Low), (High, Low), (High, High)) \}\\
    \nonumber
\end{flalign}

\label{def1}The system $\Fancy{M}$ \eqref{eq:dfsm} is secure with respect to the transitive policy $\rightarrowtail$ if $\forall \alpha \in \set{A}^*$ and domains $d \in \set{D}$, we maintain the following:
\begin{flalign}
    \lobs_u(s_0 \cdot \alpha) &= \lobs_u(s_0 \cdot \lpurge_u(\alpha))\\
    \nonumber
\end{flalign}

Intuitively, the modified system records and compares the states resulting from the two runs ($\alpha$ and $\lpurge_d(\alpha)$).  Given a system $\Fancy{M}$ \eqref{eq:dfsm}, we define the \emph{doubling construction} $\Fancy{M}^2$ as follows:
\begin{flalign}
    \Fancy{M}^2 &= \langle \set{S}', s_0', \set{A}, \lstep', \lobs', \ldom \rangle \label{em2eq}\\
    \nonumber
\end{flalign}

In this new formulation, the variations from \eqref{eq:dfsm} are defined as follows:
\begin{flalign}
    \set{S}' &= \set{S} \times \set{S}\\
    s_0' &= (s_0, s_0)\label{szerodasheq}\\
    \lstep'((s, t), a) &= \left\{
        \begin{array}{l l}
          (\lstep(s, a), \lstep(t, a)) & \text{if $\ldom(a) =$ Low}\\
          (\lstep(s, a), t) & \text{otherwise}\\
        \end{array}
        \right.\label{stepdasheq}\\
    \lobs_d'(s, t) &= (\lobs_d(s), \lobs_d(t)) \forall (s, t) \in \set{S}'\\
    \nonumber
\end{flalign}
\end{subequation}

\vskip0pt
\begin{pf*}{Proof (of \eqref{eq:q2g2}, Lemma 2)}
    First, we remind the reader of the following:
    \begin{subequation}[arabic]
    \begin{flalign}
        (s, t) \cdot a.\alpha &= \lstep((s, t), a) \cdot \alpha\\
        \nonumber
    \end{flalign}
    \end{subequation}

    So expanding $\alpha$ into its sequencec components $(a_1, a_2, \ldots a_n)$, and from \eqref{szerodasheq} and \eqref{stepdasheq}, we can derive the following:
    \begin{subequation}[arabic]
    \begin{flalign}
        \lstep'((s, t), \alpha) &= \left\{
            \begin{array}{l l}
                (\lstep(s, \alpha), \lstep(t, \alpha)) & \text{if $\ldom(\alpha) =$ Low}\\
                (\lstep(s, \alpha), t) & \text{otherwise}\\
            \end{array}
            \right.\\
        \therefore \lstep'((s_0, s_0), \alpha) &= \left\{
            \begin{array}{l l}
                (\lstep(s_0, \alpha), \lstep(s_0, \alpha)) & \text{if $\ldom(\alpha) =$ Low}\\
                (\lstep(s_0, \alpha), s_0) & \text{otherwise}\\
            \end{array}
            \right.\label{stepdasheq2}\\
        \nonumber
    \end{flalign}
    \end{subequation}

    Now given \eqref{eq:q2g1}, we can use \eqref{stepdasheq2} and derive the following:
    \begin{subequation}[arabic]
    \begin{flalign}
        s_0' \cdot \alpha &= (s, t)\\
            &= (
                \lstep(s_0, \alpha), \lstep(s_0, \alpha)
            )\\
        \therefore s &= \lstep(s_0, \alpha)\\
            &= s_0 \cdot \alpha \text{ and hence \eqref{eq:q2g2}}\\
        \nonumber
    \end{flalign}
    \end{subequation}
    \qed
\end{pf*}

\vskip0pt
\begin{pf*}{Proof (of \eqref{eq:q2g3}, Lemma 2)}
    Again using \eqref{eq:q2g1}, and \eqref{stepdasheq2}, we execute a second run but this time using the purged sequence:
    \begin{subequation}[arabic]
    \begin{flalign}
        s_0' \cdot \alpha &= (s, t)\\
            &= (
                \lstep(s_0, \lpurge_{Low}(\alpha)), \lstep(s_0, \lpurge_{Low}(\alpha))
            )\\
        \therefore t &= \lstep(s_0, \lpurge_{Low}(\alpha))\\
            &= s_0 \cdot \lpurge_{Low}(\alpha) \text{ and hence \eqref{eq:q2g3}}\\
        \nonumber
    \end{flalign}
    \end{subequation}
    \qed
\end{pf*}



\section{\question{3}{2}}
\begin{lemma}
Suppose:
\begin{subequation}[arabic]
    \begin{flalign}
        L &\leq H \text{in the order on classes}\label{eq:q2g1eq}\\
        \gamma(l) &= L\label{eq:q2g2eq}\\
        \nonumber
    \end{flalign}

    Show that it is possible to derive:
    \begin{flalign}
        \gamma &\vdash l := 0 : L_{cmd}\label{eq:q2p1}\\
        \gamma &\vdash l := 0 : !H_{cmd} \text{(Note the negation)}\label{eq:q2p2}\\
        \nonumber
    \end{flalign}
\end{subequation}
\end{lemma}

\vskip0pt
\begin{pf*}{Proof (of Lemma 3)}
    From \eqref{eq:q2g2eq} and ``VAR'' we can derive $\lgv l \colon L_{var}$, and from ``INT'' and \eqref{eq:q2p1}$^{RHS}$ we derive $\lgv 0 \colon \in \{ L, H \}$.  Using ``ASSIGN'', we can hence derive \eqref{eq:q2p1}.  However we cannot derive $\lgv l := 0 : H_{cmd}$ since the ``ASSIGN'' rule requires that the $LHS$ of the assignment be of type $H_{var}$, but there is no way to coerce a $L_{var}$ to this - hence \eqref{eq:q2p2}.
    \qed
\end{pf*}

\section{\question{4}{1}}
\begin{lemma}
    Given the following:
    \begin{flalign}
        \gamma(l) &= L\label{eq:q4g1}\\
        \gamma(h) &= H\label{eq:q4g2}\\
        \Psi &= \If h \Then l := 0 \Else l := 1\label{eq:q4g3}\\
        \nonumber
    \end{flalign}

    Show that we cannot obtain:
    \begin{flalign}
        \lgv \Psi \colon H_{cmd}\label{eq:q4p1}\\
        \nonumber
    \end{flalign}
\end{lemma}

\vskip0pt
\begin{pf*}{Proof (of Lemma 4)}
    \begin{flalign}
        \mbox{Let } &\lexpr_1 \mbox{ denote } h = 1\\
        \mbox{Let } &\lcmd_1 \mbox{ denote } l := 1\\
        \mbox{Let } &\lcmd_2 \mbox{ denote } l := 0\\
        \nonumber
    \end{flalign}

    First we analyse $\lexpr_1$; $\lexpr_1^{LHS}$ gives $H$ from ``VAR'' and ``R-VAL'', and $\lexpr_1^{RHS}$ is a numeric constant and hence gives $\tau \in \{ L, H \}$ via ``INT''.  These can be combined to give only $\tau = H$ via ``ARITH''.

    Next, we analyse $\lcmd_1$ and $\lcmd_2$ collectively as $\lcmd$ since they're identical in classification structure. $\lcmd^{RHS}$ is a numeric constant and hence gives $\tau \in \{ L, H \}$ via ``INT''.  $\lcmd^{LHS}$ is however $L_{var}$ from \eqref{eq:q4g1} and ``VAR'', so from ``ASSIGN'', we can derive only that $\lgv l := N \colon L_{cmd}$ and since $L_{cmd} \not \subseteq H_{cmd}$, it cannot be coerced as such.

    Following on from this then, it is not possible to use ``IF'' to show that $\Psi$ \emph{well-typed}, and hence \eqref{eq:q4p1} cannot be obtained.
    \qed
\end{pf*}



\section{\question{5}{7}}
\begin{subequation}[arabic]
    Suppose that following holds true:
    \begin{flalign}
        L &\leq H \text{(in the order on classes)}\label{eq:q5g1}\\
        \gamma(l) &= L\label{eq:q5g2}\\
        \gamma(h) &= H\label{eq:q5g3}\\
        \nonumber
    \end{flalign}
\end{subequation}

For each of the following programs $\Psi$, determine the set $\set{T}_\Psi$ of all types $\tau$ such that the following statement can be evaluated.
\begin{subequation}[arabic]
\begin{flalign}
    \lgv \Psi \colon \ltaucmd\label{eq:q5g4}\\
    \nonumber
\end{flalign}
\end{subequation}

\begin{lemma}
Provide a proof when $\tau \in \set{T}_\Psi$, or alternatively, an argument as to why $\tau \not\in \set{T}_\Psi$.
\end{lemma}

%\renewcommand{\labelenumi}{(\alph{enumi})}
\begin{subequation}[arabic]
From \eqref{eq:q5g2} and ``VAR'':
\begin{flalign}
    (5a) \mbox{\qmark{1}} - \Psi &= h := h + l\label{eq:q5pA}\\
    (5b) \mbox{\qmark{1}} - \Psi &= l := h + l\label{eq:q5pB}\\
    (5c) \mbox{\qmark{1}} - \Psi &= \If l \Then l := 1 \Else l := 2\label{eq:q5pC}\\
    (5d) \mbox{\qmark{1}} - \Psi &= \If h \Then h := 1 \Else h := 2\label{eq:q5pD}\\
    (5e) \mbox{\qmark{1}} - \Psi &= \While l \Do \{ h := h + l; l := l + 1 \}\label{eq:q5pE}\\
    (5f) \mbox{\qmark{1}} - \Psi &= \While h \Do \{ h := h + l; l := l + 1 \}\label{eq:q5pF}\\
    (5g) \mbox{\qmark{1}} - \Psi &= \lletvar t := 0 \lin \{ t := l; l := t; h := l \}\label{eq:q5pG}\\
    \nonumber
\end{flalign}
\end{subequation}

Before proceeding with the proofs, we will first derive some common truths which we will reference throught the seven subsections that follow.

\begin{subequation}[arabic]
First from \eqref{eq:q5g1} and ``BASE'' we have:
\begin{flalign}
    L \subseteq H\label{eq:q5d1}\\
    \nonumber
\end{flalign}

We also make an explicit note of the following fact:
\begin{flalign}
    L_{var} \not\subseteq H_{var}\label{eq:q5d2}\\
    \nonumber
\end{flalign}

From \eqref{eq:q5d1}, \eqref{eq:q5g2}, \eqref{eq:q5g3}, and ``VAR'':
\begin{flalign}
    \lgv l \colon L_{var}\label{eq:q5d3}\\
    \lgv h \colon H_{var}\label{eq:q5d4}\\
    \nonumber
\end{flalign}

Furthermore, from these equations and ``R-VAL'':
\begin{flalign}
    \lgv l \colon L\label{eq:q5d5}\\
    \lgv h \colon H\label{eq:q5d6}\\
    \nonumber
\end{flalign}

And finally, from \eqref{eq:q5d1}, \eqref{eq:q5d5}, and ``SUBTYPE'':
\begin{flalign}
    \lgv l \colon H\label{eq:q5d7}\\
    \nonumber
\end{flalign}
\end{subequation}

\vskip0pt
\begin{pf*}{Proof (of Lemma 5a)}
\begin{subequation}[arabic]
We take $\Psi^{RHS}$ from \eqref{eq:q5pA} and using \eqref{eq:q5d6}, \eqref{eq:q5d7}, and ``ARITH'' we get $\lgv \Psi^{RHS} \colon H$; that is:
\begin{flalign}
    \lgv h + l \colon H\label{eq:q5Ad1}\\
    \nonumber
\end{flalign}

Next we take $\Psi^{LHS}$ from \eqref{eq:q5pA}, \eqref{eq:q5Ad1}, \eqref{eq:q5d5}, and ``ASSIGN'' to get $\lgv \Psi : H_{cmd}$. for \eqref{eq:q5g4}.  Note that we cannot obtain $\lgv \Psi : L_{cmd}$ as the ``ASSIGN'' rule requires the $LHS$ of the assignment to be of class $L_{var}$ - which is not possible as explicitly denoted in \eqref{eq:q5d2}.  Formally then we have the following:
\begin{flalign}
    \set{T}_\Psi &= \{ H_{cmd} \}\label{eq:q5Ad2}\\
    \nonumber
\end{flalign}
\end{subequation}
    \qed
\end{pf*}

\vskip0pt
\begin{pf*}{Proof (of Lemma 5b)}
We take $\Psi^{RHS}$ \eqref{eq:q5pB}, and \eqref{eq:q5Ad1} - but cannot obtain $\lgv h + l \colon L$ as $H \not \subseteq L$. Using $\Psi^{LHS}$ from \eqref{eq:q5pB} we can only obtain \eqref{eq:q5d4}, and so we cannot apply ``ASSIGN'' to derive anything else; that is - \eqref{eq:q5pB} is not \emph{well-typed} - and hence cannot be typed.  Formally then we have the following:
\begin{subequation}[arabic]
\begin{flalign}
    \set{T}_\Psi &= \emptyset\\
    \nonumber
\end{flalign}
\end{subequation}
    \qed
\end{pf*}

\vskip0pt
\begin{pf*}{Proof (of Lemma 5c)}
We take $\Psi$ \eqref{eq:q5pC} we consider the ``IF'' rule, which is composed of three components; namely the expression $\lexpr$, the ``then'' command $\lcmd_1$, and the ``else'' command $\lcmd_2$.  Since both $\lcmd_1$ and $\lcmd_2$ are of the same class order, we will collectively refer to both as $\lcmd$.

For $\lexpr = l$ we trivially get $\lgv l \colon \tau$ where $\tau \in \{ L, H \}$ from \eqref{eq:q5d5} and \eqref{eq:q5d7}.

For the collective command $\lcmd$, we use ``ASSIGN'' and obtain $\lgv \lcmd \colon L_{cmd}$, and from ``CMD$^-$'' also obtain $\lgv \lcmd \colon H_{cmd}$.

Combining these facts with ``IF'' we hence get:
\begin{subequation}[arabic]
\begin{flalign}
    \set{T}_\Psi &= \{ L_{cmd}, H_{cmd} \}\\
    \nonumber
\end{flalign}
\end{subequation}
    \qed
\end{pf*}

\vskip0pt
\begin{pf*}{Proof (of Lemma 5d)}
We take $\Psi$ \eqref{eq:q5pD} and by the same means of evaluation as defined in the proof for Lemma 5c, and by virtue of the fact $\eqref{eq:q5d2}$, we get:
\begin{subequation}[arabic]
\begin{flalign}
    \set{T}_\Psi &= \{ H_{cmd} \}\\
    \nonumber
\end{flalign}
\end{subequation}
The cruicial point to note in this is that the ``ASSIGN'' rule requires a $LHS$ of type $\tau_{var}$, and type $\tau_{var}$ is non-ceorceable;  So as long as the $RHS$ of an assignment permits, the $LHS$ will allow only a single class type judgement to be made.
    \qed
\end{pf*}

\vskip0pt
\begin{pf*}{Proof (of Lemma 5e)}
$\Psi$ \eqref{eq:q5pE}; we consider the ``WHILE'' rule here which is composed to the two components - the expression $\lexpr$ and phrase $\lphr$.
For $\lexpr = l$ we trivially get $\lgv l \colon \tau$ where $\tau \in \{ L, H \}$ - as demonstrated earlier in proof of \eqref{eq:q5pC}.

For the phrase however, we need to expand $\lphr$ into its components first; which for this problem \eqref{eq:q5pE} can be split into the two sub-commands $\lcmd_1 = h := h + l$ and $\lcmd_2 = l := l + 1$. We know from \eqref{eq:q5Ad1} that $\lgv h + l \colon H$ (and not $L$), and from the proof of \eqref{eq:q5pC} and ``ARITH'' we can derive that $\lgv l + 1 \colon \in \{ L, H \}$.  Using the ``ASSIGN'' rule on $\lcmd_1$ and $\lcmd_2$ then we get $\lgv \lcmd_1 \colon L_{cmd}$, and $\lgv \lcmd_2 \colon \{ L_{cmd}, H_{cmd} \}$ respectively.  From ``COMPOSE'' we know that $H_{cmd} \subseteq L_{cmd}$, and so we can safely class this problem as follows:
\begin{subequation}[arabic]
\begin{flalign}
    \set{T}_\Psi &= \{ L_{cmd}, H_{cmd} \}\\
    \nonumber
\end{flalign}
\end{subequation}
    \qed
\end{pf*}

\vskip0pt
\begin{pf*}{Proof (of Lemma 5f)}
$\Psi$ \eqref{eq:q5pF}; we consider the ``WHILE'' rule here which is composed to the two components - the expression $\lexpr$ and phrase $\lphr$, as in the previous proof.
For $\lexpr = l$ we trivially get $\lgv l \colon \tau$ where $\tau \in \{ L, H \}$ - as demonstrated earlier in proof of \eqref{eq:q5pC}.
This time however, we cannot coerce $\lcmd_1$ to $L$, because $L_{cmd} \not \subseteq H_{cmd}$;  That is, this problem is not \emph{well-typed}.  Formally then:
\begin{subequation}[arabic]
\begin{flalign}
    \set{T}_\Psi &= \emptyset\\
    \nonumber
\end{flalign}
\end{subequation}
    \qed
\end{pf*}

%\vskip0pt
%\begin{pf*}{Proof (of Lemma 5g)}
%\begin{subequation}[arabic]
%\begin{flalign}
    %\nonumber
%\end{flalign}
%\end{subequation}
%\end{pf*}

References: \cite{vdML5b10}, \cite{SY08}, and \cite{MD003}.

%-------------------------------------------------------------------------------
%-------------------------------------------------------------------------------
\pagebreak
\IfFileExists{local.bib}{
    \bibliography{autonomy,local}
}{
    \bibliography{autonomy}
}

%%%%%%%%%%%%%%%%%%%%%%%%%%%%%%%%%%%%%%%%%%%%%%%%%%%%%%%%%%%%%%%%%%%%%%%%%%%%%%%%
\end{article}
\bibliographystyle{alpha}
\end{document}
\endofdocument

