%!TEX TS-program = LaTeX
%!TEX encoding = UTF-8 Unicode
\documentclass[namedreferences]{autons}

\newcommand{\theId}{0x003}
\newcommand{\theVersion}{0.1}
\newcommand{\theTitle}{Static Program Analysis}
\newcommand{\theKeywords}{security}
\newcommand{\theAbstract}{
    Static analysis aims to show that a program is safe in some well-defined way, for example - that it is secure with respect to some well-understood definition of the word `security'.  It does so purely on an analysis of the code, and independent of any inputs.
}

%%%%%%%%%%%%%%%%%%%%%%%%%%%%%%%%%%%%%%%%%%%%%%%%%%%%%%%%%%%%%%%%%%%%%%%%%%%%%%%%
%. The Preamble...
\def\copyrightline{\copyright\- Nima Talebi 2010}
\newcommand{\theFirstName}{Nima}
\newcommand{\theLastName}{Talebi}
\newcommand{\theEmail}{nima@autonomy.net.au}
\newcommand{\theSubtitle}{Microdocs - Article {\tt\theId}}
\newcommand{\theInstitute}{Autonomy Corporation Pty Ltd}
\newcommand\theAuthor{\theFirstName \theLastName}
%-------------------------------------------------------------------------------
\usepackage[usenames,dvipsnames]{color}
\usepackage[
    pdftitle={\theTitle},
    pdfsubject={\theSubtitle},
    pdfauthor={\theAuthor},
    pdfcreator={\theAuthor},
    pdfkeywords={\theKeywords},
    backref,
    hyperfigures=true,
    final,
    bookmarks=false,
    xetex,
    dvipdfm,
    breaklinks,
    bookmarksopen=true,
    bookmarksnumbered=true,
    colorlinks=true,
    linkcolor=Cyan,
    anchorcolor=Red,
    citecolor=ForestGreen,
    filecolor=Aquamarine,
    menucolor=Orange,
    runcolor=Red,
    urlcolor=LimeGreen
]{hyperref}
\InputIfFileExists{preamble.tex}{}{}

%%%%%%%%%%%%%%%%%%%%%%%%%%%%%%%%%%%%%%%%%%%%%%%%%%%%%%%%%%%%%%%%%%%%%%%%%%%%%%%%
%. The Document...
\begin{document}
\begin{article}
\begin{opening}
    \title{\theTitle}
    \runningtitle{\theTitle}
    \subtitle{\theSubtitle}

    \runningauthor{\theAuthor}
    \author{\theFirstName\ \surname{\theLastName}\email{\theEmail}}
    \institute{\theInstitute}
    \InputIfFileExists{authors.tex}{}{}

    \IfFileExists{motto.tex}{
        \begin{motto}[prose]
            The only thing that never looks right is a rule. There is not in existence a page with a rule on it that cannot be instantly and obviously improved by taking the rule out.  \rightline{George Bernard Shaw, In {\it The Dolphin} (1940)}


        \end{motto}
    }{}

    \begin{abstract}
        \theAbstract
    \end{abstract}
    \keywords{\theKeywords}

    \InputIfFileExists{abbrev.tex}{}{}
    \InputIfFileExists{nomen.tex}{}{}
    \InputIfFileExists{dedication.tex}{}{}
    \InputIfFileExists{classify.tex}{}{}
    \classification{Microdoc}{\theId}
    \classification{Version}{\theVersion}
\end{opening}

%-------------------------------------------------------------------------------
%. Finish off...
\addtocounter{tocdepth}{1}
\tableofcontents


\newcommand\taucmd{\tau_\texttt{cmd}}
\newcommand\tauvar{\tau_\texttt{var}}
\newcommand\letvar{\texttt{letvar}}
\newcommand\cmd{\varsigma}
\newcommand\expr{\varepsilon}
\newcommand\phr{\wp}
%-------------------------------------------------------------------------------

\section{Notation}

\begin{flalign}
    \leq & \colon \mbox{the \emph{partial order relation}}\\
    \subseteq &\colon \mbox{the \emph{substype relation} - partial order relation extension}\\
    (SC, \leq) &\colon \mbox{the lattice that defines an information flow policy}\label{SCeq1}\\
    SC &\colon \mbox{a finite set of \emph{security classes} partially ordered}\label{SCeq2}\\
    \gamma &\colon \mbox{identifier typing}\\
    \tau &\colon \mbox{data types } \in SC\\
    \taucmd &\colon \mbox{command types}\\
    \tauvar &\colon \mbox{variable types storing information whose s.c. is $\leq\tau$ }\\
    \rho &\colon \mbox{phrase types} - \mbox{ includ data types }
        \tau \in \{ \expr, \tauvar, \taucmd \}\\
    \expr &\colon \mbox{an expression}
        \in \{ x, n, \expr \sim \expr'\} \mbox{ where } \sim \in \{ +, -, <, =\}\\
    \cmd &\colon \mbox{a command}
        \in \{\nonumber\\
            &\HSpace \expr := \expr', \cmd; \cmd',\nonumber\\
            &\HSpace \mbox{\texttt{if}-construct},\nonumber\\
            &\HSpace \mbox{\texttt{while}-construct},\nonumber\\
            &\HSpace \mbox{\texttt{letvar}-construct}\nonumber\\
        \}\\
    \phr &\colon \mbox{a phrase } \in \{ \expr, \cmd \}\\
    x &\colon \mbox{a variable}\\
    n &\colon \mbox{an integer constant}\\
    \gamma \vdash \phr \colon \tau &\colon \mbox{a judgement on $\phr$ w.r.t. $\gamma$}\\
        &\mbox{asserting the type of $\phr$ to be $\tau$}\nonumber\\
        &\mbox{(i.e., $\gamma\colon Variables \rightarrow SC$)}\nonumber\\
    \gamma \vdash \phr \colon \rho &\colon \mbox{a judgement on $\phr$ w.r.t. $\gamma$}\\
        &\mbox{asserting the phrase $\phr$ is of type $\rho$}\nonumber\\
        &\mbox{(i.e., $\gamma\colon Variables \rightarrow SC$)}\nonumber\\
    \nonumber
\end{flalign}

\section{Definitions}
From the \eqref{SCeq1} and \eqref{SCeq2} above then, it is obvious that we can concretely define $SC$ as a finit set of security classes, partially ordered by the partial binary relation ``$\leq$''.

\DEFN{Certificate Condition}{
    Every program construct has a \emph{certificate condition} which is a purely syntactic condition relating security classes.  Some of these control \emph{explicit flows}, while others control \emph{implicit flows} \cite{VIS96}.  The following example demonstrates both types:\\
    \\
    \begin{tabular}{ | l | l | }
        \hline
        Program & Control Type \\
        \hline
        \texttt{y := x} & Explicit\\
        \hline
        \texttt{\textbf{if} x \textbf{then} y := 1 \textbf{else} y = 0} & Implicit\\
        \hline
    \end{tabular}
}

\section{Rules of Inference}
\subsection{Typing and Subtyping Rules}

The following set of rules is based on \cite{VIS96}:
\begin{flalign}
    BASE &\colon \frac{\tau \leq \tau'}{\tau \subseteq \tau'}\label{BASE}\\
    REFLEX &\colon \rho \subseteq \rho'\\
    TRANS &\colon \frac{\rho \subseteq \rho' \wedge \rho' \subseteq \rho''}{\rho \subseteq \rho''}\\
    CMD^- &\colon \frac{\tau \subseteq \tau'}{\taucmd' \subseteq \taucmd}\label{CMD}\\
    SUBTYPE &\colon \frac{
        \gamma \vdash \phr \colon \rho \wedge \rho \subseteq \rho'
    }{
        \gamma \vdash \phr \colon \rho'
    }\\
    INT &\colon \gamma \vdash N \colon \tau \quad\mbox{where $N$ is an integer constant}\\
    VAR &\colon \gamma \vdash x \colon \tauvar \quad\mbox{if } \gamma(x) = \tau\\
    ARITH &\colon \frac{
        \gamma \vdash \expr \colon \tau \wedge \gamma \vdash \expr' \colon \tau
    }{
        \gamma \vdash \expr \sim \expr' \colon \tau
    }\\
        &\mbox{where $\sim$ is any binary arithmatic operator}\nonumber\\
    R-VAL &\colon \frac{
        \gamma \vdash \expr \colon \tauvar
    }{
        \gamma \vdash \expr \colon \tau
    }\\
    ASSIGN &\colon \frac{
        \gamma \vdash x \colon \tauvar \wedge \gamma \vdash \expr \colon \tau
    }{
        \gamma \vdash x := \expr \colon \taucmd
    }\\
    COMPOSE &\colon \frac{
        \gamma \vdash \cmd \colon \taucmd \wedge \gamma \vdash \cmd' \colon \taucmd
    }{
        \gamma \vdash \cmd;\cmd' := \colon \taucmd
    }\\
    IF &\colon \frac{
        \gamma \vdash \expr \colon \tau
            \wedge \gamma \vdash \phr \colon \taucmd
            \wedge \gamma \vdash \phr' \colon \taucmd
    }{
        \gamma \vdash \texttt{if } \expr \texttt{ then } \phr \texttt{ else } \phr' \colon \taucmd
    }\\
    WHILE &\colon \frac{
        \gamma \vdash \expr \colon \tau \wedge \gamma \vdash \phr \colon \taucmd
    }{
        \gamma \vdash \texttt{while } \expr \texttt{ do } \phr \colon \taucmd
    }\\
    LETVAR &\colon \frac{
        \gamma \vdash \expr \colon \tau \wedge \gamma[x\colon\tau] \vdash \cmd \colon \taucmd'
    }{
        \gamma \vdash \texttt{letvar } x := \expr \texttt{ in } \cmd \colon \taucmd'
    }\\
    \gamma[x\colon\tau] &\colon \mbox{ where } \left\{
        \begin{array}{l l}
            \gamma'(y) &= \gamma(y) \mbox{ if } y \neq x\\
            \gamma'(x) &= \tau\\
        \end{array}\right.\\
    \nonumber
\end{flalign}

\NOTE{There is NO subtyping rule for $\tauvar$, that is - this expression always tells the \emph{actual} type of the variable}

\DSMB{Coercion}{
    Drop of command or class type in special circumstances as detailed below.
}

Using \eqref{BASE} and \eqref{CMD}, two types of coercions\cite{VIS96} that are possible are as follows:
\begin{flalign}
    \mbox{if } \tau \subseteq \tau', \mbox{ then } \left\{
        \begin{array}{l l}
            \tau \rightarrow \tau' &\quad \mbox{Data Type Coercion}\\
            \taucmd' \rightarrow \taucmd &\quad \mbox{Command Type Coercion}\\
        \end{array}\right.\label{coerceeq}\\
    \nonumber
\end{flalign}

%-------------------------------------------------------------------------------
%-------------------------------------------------------------------------------
\pagebreak
\IfFileExists{local.bib}{
    \bibliography{autonomy,local}
}{
    \bibliography{autonomy}
}

%%%%%%%%%%%%%%%%%%%%%%%%%%%%%%%%%%%%%%%%%%%%%%%%%%%%%%%%%%%%%%%%%%%%%%%%%%%%%%%%
\end{article}
\bibliographystyle{alpha}
\end{document}
\endofdocument

