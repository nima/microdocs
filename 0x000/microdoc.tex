\documentclass{microdoc}

\title{The Quadratic Formula}
\author{\authab{Robert Smith}{Quad's Printing}{quadricode@gmail.com}\\
        \autha{Nima Talebi}{Autonomy Corporation Pty Ltd}}
\docnum{0x000}
\version{0.1}
\keywords{quadratic formula, polynomials, roots}
\prereqs{A basic knowledge of algebra is needed as well as the method
  of completing the square.}
\abstract{The so called \emph{quadractic formula} is an expression
  able to compute the roots of a second-degree polynomial in
  essentially $O(1)$ time.}

\begin{document}
\maketitle

\section{Introduction}
The basic premise of this microdoc is to derive that given a monic
polynomial\footnote{A \emph{monic polynomial} is a polynomial whose
  leading coefficient is $1$.} $f(x)=x^2+bx+c$, the values of $x$ such
that $f(x)=0$ are \[x_{1,2}= \frac{-b\pm\sqrt{b^2-4c}}{2}.\] This is
typically learned in the sixth or seventh year of school and proves to
be an invaluable tool in all fields of science and mathematics, for it
can compute things from trajectory solutions to factorizations.

\subsection{A Computer Implementation}
We can implement such a function in the C programming language
easily. Since C cannot return two values in a native fashion, we must
indicate which solution we want via a number \verb|soln|. We also
specify the \emph{discriminant}, $b^2-4c$, which signifies how many
roots of $f(x)$ exist, if any. Let $D$ be the discriminant. Then in general,
\begin{itemize}
  \item If $D>0$, there are two real roots,
  \item if $D=0$, there is exactly one real root,
  \item if $D<0$, there are no real roots.
\end{itemize}
Note that the case when $D=0$, we have
\begin{align*}
  \frac{-b\pm\sqrt{b^2-4c}}{2} &= \frac{-b\pm\sqrt{D}}{2}\\
                               &= \frac{-b\pm\sqrt{0}}{2}\\
                               &= \frac{-b\pm 0}{2}\\
                               &= \frac{-b}{2},
\end{align*}
so the plus-or-minus vanishes. If $D<0$, clearly we have the root of a
negative number, which requires the use of another class of numbers
called \emph{imaginary numbers}\footnote{Essentially, we define
  $\sqrt{-1} = i$, where $i$ is called the \emph{imaginary unit}, and
  so $\sqrt{-D}$, where $D>0$, is equivalent to the imaginary number
  $\sqrt{-1}\sqrt{D}=i\sqrt{D}$.}

The implementation in figure \ref{fig:quadC} does not use imaginary
numbers, for the implementation would require extra definitions.
\begin{figure}[h]
\centering
\begin{lstlisting}[language=C,basicstyle=\ttfamily\footnotesize]
float quadratic_formula(float b, float c, int soln)
{
    float discriminant;
    discriminant = b*b - 4.0*c;
    if (1 == soln)
        return (-b + sqrt(discriminant))/2.0;
    else
        return (-b - sqrt(discriminant))/2.0;
}
\end{lstlisting}
\caption{Implementation of the quadratic formula.}
\label{fig:quadC}
\end{figure}
\end{document}